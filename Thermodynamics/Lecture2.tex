\section{Relevant Mathematics}

\subsection{Total differentials}
$$
	\mathrm{d}Z = \left(\frac{\partial z}{x}\right)_y \mathrm{d}x +
	\left(\frac{\partial z}{\partial y}\right)_x \mathrm{d} y
$$
This lets us see what happens to the system when we make a small change to
some variable(s). For large changes, it is best to add up small increments
(i.e.\ integrate):
$$
	\Delta Q = \int C_p \cdot \mathrm{d}T
$$
In mathematics, partial derivatives are relatively straightforward. However,
here, what about:
$$
	\left(\frac{\partial U}{\partial V}\right)_p?
$$
\paragraph{Example:} Find the above:
$$
	U = \frac{1}{2}mv^2 = \frac{3}{2}Nk_BT = \frac{3}{2}pV
$$
Now, if we differentiate:
$$
	\left(\frac{\partial U}{\partial V}\right)_T =
	\frac{\partial}{\partial V}\left[\frac{3}{2}Nk_b T\right] = 0
$$
$$
	\left(\frac{\partial U}{\partial V}\right)_p = 
	\frac{\partial}{\partial V}\left[\frac{3}{2}pV\right] = \frac{3}{2}p
$$
It is important to remember which variable you are keeping constant in
thermodynamics, or things can get out of hand quickly.

\subsection{Some Relations}
If we have that $x$, $y$ and $z$ are related by $f(x,y,z)$, then we also have
that $x = x(y,z)$, $y = y(x,z)$.

\subsubsection{Useful properties of 3D functions}
There are two useful things to use:
$$
	\left(\frac{\partial x}{\partial y}\right)_z
	\left(\frac{\partial y}{\partial x}\right)_z = 1
$$
$$
	\left(\frac{\partial x}{\partial y}\right)_z
	\left(\frac{\partial y}{\partial z}\right)_x
	\left(\frac{\partial z}{\partial x}\right)_y = -1
$$
\begin{proof}
	Begin with the original partial above:
	$$
		\mathrm{d}x = \left(\frac{\partial x}{\partial y}\right)_z \mathrm{d}y
		+ \left(\frac{\partial x}{\partial z}\right)_y \mathrm{d}z
	$$
	We can also repeat it with a change of variables to $y$:
	$$
		\mathrm{d}y = \left(\frac{\partial y}{\partial x}\right)_z \mathrm{d}x
		+ \left(\frac{\partial y}{\partial z}\right)_x \mathrm{d}z
	$$
	Substituting:
	$$
		\mathrm{d}x = \left(\frac{\partial x}{\partial y}\right)_z
		\left[
			\left(\frac{\partial y}{\partial x}\right)_z \mathrm{d}x +
			\left(\frac{\partial y}{\partial z}\right)_x \mathrm{d}z
		\right] + \left(\frac{\partial x}{\partial z}\right)_y \mathrm{d}z
	$$
	Which we can multiply out to:
	$$
		\mathrm{d}x = \left(\frac{\partial x}{\partial y}\right)_z
		\left(\frac{\partial y}{\partial x}\right)_z \mathrm{d}x
		+ 
		\left[
			\left(\frac{\partial x}{\partial y}\right)_z
			\left(\frac{\partial y}{\partial z}\right)_x +
			\left(\frac{\partial x}{\partial z}\right)_y
		\right] \mathrm{d}z
	$$
	And comparing coefficients, we must see that the right $=0$ and the left
	$=1$, giving:
	$$
	    \left(\frac{\partial x}{\partial y}\right)_z
		\left(\frac{\partial y}{\partial x}\right)_z = 1
	$$
	$$
		\left(\frac{\partial x}{\partial y}\right)_z
		\left(\frac{\partial y}{\partial z}\right)_x
		\left(\frac{\partial z}{\partial x}\right)_y = -1
	$$
\end{proof}

\paragraph{Example:} Total differential of an ideal gas.

If we consider 1 mole, we have:
$$
	pV = RT
$$
To differentiate:
$$
	\mathrm{d}V = \left(\frac{\partial V}{\partial T}\right)_p \mathrm{d} t +
	\left(\frac{\partial V}{\partial p}\right)_T \mathrm{d}p
$$
We can now work out the compoenents, and substitute in:
$$
	\mathrm{d}V = \frac{RT}{p}\left[\frac{\mathrm{d}T}{T} - 
	\frac{\mathrm{d}p}{p}\right]
$$
And finally:
$$
	\frac{\mathrm{d}V}{V} = \frac{\mathrm{d}T}{T} - \frac{\mathrm{d}p}{p}
$$
This tells us a small change in temperature or pressure leads to a small change
in volume - which is what we would expect.

\section{Exact and inexact differentials}
With well behaved functions, the order of differentiation doesn't matter. These
are called exact differentials.
\paragraph{Exact differentials} include system properties, for example
temperature or pressre.
\paragraph{Inexact differentials} are things that aren't in the function of
state, for example work - that are path dependent.

\subsection{Changes in variables}
We can say that an incremental volume change d$V$ gives a total volume change
between states 1 and 2:
$$
	\int^2_1 = V_2 - V_1 = \Delta V
$$
This doesn't take into account anything that happens in the middle because it
doesn't matter.

However, things like work;
$$
	W_{1\rightarrow 2} = \int^2_1 \delta W \neq W_2 - W_1
$$
This is because the work done is the area under the $pV$ graph and as such
depends on the path taken between two points.

\section{Changes in notation}
In the previous course, work in was negative and work out was positive. Here:
\paragraph{WORK IN IS POSITIVE}
\paragraph{WORK OUT IS NEGATIVE}

\section{Work-energy}
Work and energy are linked - work is equivalent to raising a weight (Joule).
Energy contained in a system is its capacity to do work.

Two things not in thermal equilibrium connected via an engine get work out.

\section{First Law of Thermodynamics}
Internal energy ($U$) is the sum of all of the internal degrees of freedom,
which ties into the First Law.

The work done by adiabatic paths is the same for all paths. Experiments show
that when a system changes between states by differnet adiabatic paths, as
measured by a change in the level of a weight, the work done is always the
same.

This law is based on empirical evidence and is a statement regarding the
conservation of energy. Mathematically:
$$
	\mathrm{d}U = \delta Q + \delta W
$$
(here $\delta$ is used to signify a non-exact differential).

\subsection{Heat Capacity}
Heat capacties describe how heat changes the internal energy of a gas. We have
that:
$$
	C_V = \left(\frac{\partial U}{\partial T}\right)_V
$$
$$
	C_p - C_V = R
$$
\begin{proof}
Begin with:
$$
	U = U(V,T)
$$
$$
	\mathrm{d}U = \left(\frac{\partial U}{\partial V}\right)_T \mathrm{d}V +
	\left(\frac{\partial U}{\partial T}\right)_V \mathrm{d}T
$$
And we find by using the ideal gas law:
$$
	\mathrm{d}U = \delta Q - p \delta V
$$
Eliminate $\mathrm{d}U$:
$$
	\delta Q = \left(\frac{\partial U}{\partial T}\right)_V \mathrm{d}T + 
	\left[
		\left(\frac{\partial U}{\partial V}\right)_T + p
	\right] \mathrm{d}V
$$
Dividing through by d$T$ and realising at constant volume d$V = 0$:
$$
	\left(\frac{\partial Q}{\partial T}\right)_V = C_V \;(\mathrm{def})\; =
	\left(\frac{\partial U}{\partial T}\right)_V
$$
We do a similar thing for $C_p$:
$$
	C_p = \left(\frac{\partial U}{\partial T}\right)_V + 
	\left[
		\left(\frac{\partial U}{\partial V}\right)_T + p
	\right] \left(\frac{\partial V}{\partial T}\right)_p
$$
Rearranging, we find that:
$$
	C_p - C_V = \left[
	\left(\frac{\partial U}{\partial V}\right)_T + p
	\right] \left(\frac{\partial V}{\partial T}\right)_p
$$
And from the ideal gas law, we know that $U = U(T)$ and as such:
$$
	C_p - C_V = R
$$
\end{proof}

Physical laws are reversible. Something is thermodynamically reversible if its
direction can be changed by an infinitesimal change to a system property.
\paragraph{Quasistatic} means that we have no shockwaves, not heat transfer
etc. and this is the way that we study thermodynamics - in a `perfect'
environment.

In real processes energy will be converted to heat and dissapated.

\pargraph{Example: }Heat is taken in for an isothermal expansion.
\begin{proof}
Isothermal means that d$T=0$, and for an ideal gas $U = U(T)$. We have:
$$
	\mathrm{d}U = \delta Q + \delta W
$$
Because there is no temperature change:
$$
	\delta Q = - \delta W
$$
$$
	\delta Q = p \delta V
$$
And from this we can find:
$$
	\Delta Q = \int_{V_1}^{V_2} p \mathrm{d}V = 
$$
$$
	\int_{V_1}^{V_2} \frac{RT}{V}\cdot\mathrm{d}V = 
	RT\ln\left(\frac{V_2}{V_1}\right)
$$
And as such, if $V_2 > V_1$ this implies $\Delta Q > 0$.
\end{proof}

\paragraph{Example: }$pV^\gamma$ is constant for an adiabatic process on an
ideal gas.
\begin{proof}
Start with the 1$^{st}$ law:
$$
	\mathrm{d}U = \delta Q + \delta W
$$
On an adiabatic process, $\delta Q = 0$:
$$
	\mathrm{d}U = - p \mathrm{d}V
$$
We also have that for an ideal gas:
$$
	C_V = \left(\frac{\partial Q}{\partial T}\right)_V = 
	\left(\frac{\partial U}{\partial T}\right)_V
$$
Substituting in:
$$
	\mathrm{d}T = -p \frac{\mathrm{d}V}{C_V}
$$


\end{proof}
